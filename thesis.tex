\documentclass{viser-thesis}

\author{Mateja Marić}
\indeks{NRT-12/19}
\studijskiProgram{Nove računarske tehnologije}
\mentor{Dr Perica Štrbac}
\predmet{Funkcionalno programiranje}
\title{Primer funkcionalnog programiranja Veb aplikacija u programskom jeziku Elm}
\date{septembar 2025.}
\prviZadatak{Koncepti funkcionalnog programiranja}
\drugiZadatak{Opis programskog jezika Elm}
\treciZadatak{Implementacija Veb aplikacija u programskom jeziku Elm}

\begin{document}

\maketitle
\pagenumbering{Roman}
\infopage
{
\pagestyle{plain}
\tableofcontents
\newpage
\pagestyle{fancy}% Turn on the style (for header and footer)
}

\pagenumbering{arabic}% Arabic page numbers (and reset to 1)

\section{Uvodna reč autora}
\newpage

\section{Koncepti funkcionalnog programiranja}
Knjiga \textcite{feldman2020elm} je fantasticna!
U funkcionalnom programiranju se cesto koristi flatMap\index{flatMap}, sto je do jaja! \parencite{feldman2020elm}
\newpage

\subsection{Core Principles}

\subsubsection{Immutability}

Data structures cannot be changed once created. Updates create new versions instead.

\subsubsection{First-class functions}

Functions are treated as values: they can be passed as arguments, returned, or stored in data structures.

\subsubsection{Higher-order functions}

Functions that take other functions as arguments or return them.

\subsubsection{No side effects}

Functions do not alter external state (no mutation of variables, I/O, or global state inside function logic).

\subsubsection{Referential transparency}

An expression can be replaced by its value without changing the program’s behavior.

\subsubsection{Declarative style}

Programs describe *what* to compute, not *how* to compute it step by step.

\newpage

\subsection{Functional Abstractions}

\subsubsection{Pure functions}

Output depends only on input, with no hidden state or external dependencies.

\subsubsection{Recursion instead of loops}

Iteration is achieved by recursive functions, since mutable state and loop counters are avoided.

\subsubsection{Lazy evaluation}

Values are computed only when needed (common in purely functional languages like Haskell).

\subsubsection{Pattern matching}

Directly destructuring and handling different data shapes (like algebraic data types).

\subsubsection{Type systems and algebraic data types (ADTs)}

Strong, expressive type systems with `sum types` (variants) and `product types` (records/tuples).

\subsubsection{Currying and partial application}

Functions naturally take one argument; multi-argument functions are built from single-argument ones.

\subsubsection{Function composition}

Building complex functions by combining smaller ones.

\newpage

\subsection{Managing Effects in PFP}

\subsubsection{Monads}

Structures to model sequential computations and manage effects (e.g., `Maybe`, `IO`, `State`).

\subsubsection{Functors and Applicatives}

Abstractions for mapping and combining computations in a pure way.

\subsubsection{Persistent data structures}

Efficient immutable collections that share structure between versions.

\subsubsection{Controlled side effects via purity}

Side effects (I/O, randomness, state) are modeled explicitly, not hidden inside functions.

\newpage

\subsection{Conceptual Benefits}

\subsubsection{Equational reasoning}

Programs can be reasoned about like algebraic equations due to referential transparency.

\subsubsection{Parallelism and concurrency safety}

No shared mutable state → easier to parallelize computations without race conditions.

\subsubsection{Determinism}

Same inputs always give the same outputs.

\newpage

\section{Opis programskog jezika Elm}
\newpage
\subsection{Druga sekcija prva podsekcija}
\newpage
\subsubsection{Podpodsekcija}
\newpage
\subsection{Druga sekcija druga podsekcija}
\newpage
\subsubsection{Podpodsekcija}
\newpage
\subsection{Druga sekcija treca podsekcija}
\newpage
\subsubsection{Podpodsekcija}
\newpage

\section{Implementacija Veb aplikacija u programskom jeziku Elm}
\newpage
\subsection{Treca sekcija prva podsekcija}
\newpage
\subsubsection{Podpodsekcija}
\newpage
\subsection{Treca sekcija druga podsekcija}
\newpage
\subsubsection{Podpodsekcija}
\newpage
\subsection{Treca sekcija treca podsekcija}
\newpage
\subsubsection{Podpodsekcija}
\newpage

\printindex
\newpage

\printbibliography[heading=bibintoc]

\end{document}

% Пример функционалног програмирања Веб апликације у програмском језику Elm
% Код ментора: др Перица Штрбац, дипл. инж.

% Основни задаци завршног рада:
% 1. Концепти функционалног програмирања.
% 2. Опис програмског језика Elm.
% 3. Имплементација Веб апликације у програмском језику Elm.

% Предмет из кога се ради завршни рад: Функционално програмирање

\documentclass{viser-thesis}

\author{Mateja Marić}
\title{Primer funkcionalnog programiranja Veb aplikacija u programskom jeziku Elm}
\date{septembar 2023.}

\begin{document}

\maketitle
\tableofcontents
\newpage

\section{Prva sekcija}
\newpage
\subsection{Prva sekcija prva podsekcija}
\newpage
\subsubsection{Podpodsekcija}
\newpage
\subsection{Prva sekcija druga podsekcija}
\newpage
\subsubsection{Podpodsekcija}
\newpage
\subsection{Prva sekcija treca podsekcija}
\newpage
\subsubsection{Podpodsekcija}
\newpage

\section{Druga sekcija}
\newpage
\subsection{Druga sekcija prva podsekcija}
\newpage
\subsubsection{Podpodsekcija}
\newpage
\subsection{Druga sekcija druga podsekcija}
\newpage
\subsubsection{Podpodsekcija}
\newpage
\subsection{Druga sekcija treca podsekcija}
\newpage
\subsubsection{Podpodsekcija}
\newpage

\section{Treca sekcija}
\newpage
\subsection{Treca sekcija prva podsekcija}
\newpage
\subsubsection{Podpodsekcija}
\newpage
\subsection{Treca sekcija druga podsekcija}
\newpage
\subsubsection{Podpodsekcija}
\newpage
\subsection{Treca sekcija treca podsekcija}
\newpage
\subsubsection{Podpodsekcija}

\end{document}
